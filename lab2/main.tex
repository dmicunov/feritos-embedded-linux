\documentclass[11pt]{article}
\usepackage[croatian]{babel}  % Croatian typographical rules and hyphenation patterns
\usepackage[utf8]{inputenc}  	% Encoding of Croatian characters
\usepackage[T1]{fontenc}
\usepackage{ae,aecompl}     	% Type 1 fonts, similar to Computer Modern

\usepackage{microtype}				% Improves spacing

\usepackage{booktabs}					% Nice looking tables
\usepackage{enumerate}				% Additional options for listing of items in enumerate environment
\usepackage{algorithm2e}			% Writing pseudo-code
\usepackage{todonotes}				% Adding todo items
\usepackage{dirtree}					% Simple display of directory tree
\usepackage{hyperref}
\usepackage{graphicx}
\usepackage{subfig}
\usepackage{caption}
\usepackage{listings}

\graphicspath{ {./images/} }

\hypersetup{
    colorlinks=true,
    linkcolor=blue,
    filecolor=magenta,
    urlcolor=cyan,
}
\urlstyle{same}
\usepackage{graphics}
\title{
	\Large Sveučilište Josipa Jurja Strossmayera u Osijeku \\
	Fakultet elektrotehnike, računarstva i informacijskih tehnologija \\
	\vspace{4cm}
	\Large Kolegij: Linux u ugradbenim sustavima \\
	\vspace{4cm}
	\Large \textbf{Laboratorijska vježba 2: Upoznavanje s Raspberry Pi 3
	računalom. Izgradnja Linuxa za Raspberry Pi 3}
	}
\date{}
\begin{document}
\maketitle
\thispagestyle{empty}
\newpage

\section{Uvod}
Na tržištu se početkom 2012. godine pojavilo računalo Raspberry Pi i pokrenulo
 malu revoluciju u području obrazovanja i hobija vezanih za računalne znanosti.
 Ovo računalo je postalo izrazito popularno zbog svoje cijene (oko 35\$) i
 malih dimenzija (8,6cm x 5,4cm x 1,7cm). Iako je Raspberry Pi računalo opće
 namjene, ima mogućnost priključivanja različite opreme (npr. različiti senzori
 i aktuatori) pa se koristi u mnogim hobi projektima u području ugradbenih
 računalnih sustava i interneta objekata, a sve češće i u različitim
 komercijalnim projektima. Glavni dio ovog računala predstavlja SoC (engl.
 \textit{System On a Chip}) uz koji se nalazi različita periferija poput RAM
 memorije, HDMI izlaza, Ethernet priključka i sl. Također, na pločici se
 nalaze i pinovi koji predstavljaju ulazno izlazne pinove opće namjene (engl.
 \textit{General Purpose Input Output - GPIO}) koji omogućuje spajanje, u
 svijetu PC-a, nestandardne opreme.

\section{Modeli Raspberry Pi računala}
Do danas je na tržište izašlo nekoliko generacija Raspberry Pi računala.
Svima je zajednički Broadcom SoC s integriranim ARM procesorom (CPU) i
 ugrađenim grafičkim procesorom (GPU).
Brzina takta procesora se kreće od 700 MHz do 1.4 GHz kod modela Pi 3B+ ili
 1.5 GHz kod modela Pi 4. Količina memorije (RAM) se kreće od 256 MB do 1 GB
 ili čak 4 GB kod najjačeg modela najnovijeg Pi 4. \textit{Secure Digital} (SD)
 kartice se koriste za pohranu operativnog sustava i programa. Na pločama se
 nalaze do četiri USB priključka. Za video izlaz koristi se HDMI.
 Pomoću GPIO pinova podržani su standardni protokoli kao što je
 I\textsuperscript{2}C.
Svi B modeli imaju \textit{Ethernet} priključak, dok modeli Pi 3 i Pi Zero W
imaju \textit{Wi-Fi 802.11n} te \textit{Bluetooth}.
\newline
\newline
\textbf{Raspberry Pi Zero} manjih dimenzija i s reduciranim brojem
 ulazno-izlaznih jedinica te ulazno-izlaznih jedinica opće namjene (GPIO) je
 izašao na tržište 2015. godine s cijenom od 5\$. 2017. godine izlazi
 \textbf{Raspberry Pi Zero W} s dodanim \textit{Wi-Fi} i \textit{Bluetooth}
 mogućnostima.
\begin{figure}[h!]
\centering
\includegraphics[width=0.8\textwidth]{rpi-zero.jpg}
\captionsetup{justification=centering}
\caption{Raspberry Pi Zero}
\end{figure}
\newline
\textbf{Raspberry Pi 3 Model B} je izašao 2016. godine s 1.2 GHz 64-bitnim
 četverojezgrenim procesorom te ugrađenim 802.11n Wi-Fi i Bluetooth mrežnim
 mogućnostima. 2018. godine izlazi \textbf{Raspberry Pi 3 Model B+} s 1.4 GHz
 procesorom te gigabitnom \textit{Ethernet} mrežom ne punih mogućnosti (u
 stvarnim testovima dostiže brzinu 300 Mbit/s, jer dijeli USB 2.0 sabirnicu).
 Posjeduje \textit{dual-band 802.11ac 2.4/5 Ghz Wi-Fi} bežični mrežni standard.
\newline
 Najnoviji model izlazi 2019. godine. Radi se o modelu \textit{Raspberry Pi
 4 Model B}. Hardverski definitivno najjači model Raspberry Pi računala s
 1.5 GHz četverojezgrenim \textbf{ARM Cortex-A72} procesorom, podrškom za
 802.11ac Wi-Fi bežičnom mrežom, \textbf{Bluetooth 5} podrškom,
 podrškom za gigabitnom Ethernet mrežom punih mogućnosti, dva \textbf{USB 2.0}
 porta, dva \textbf{USB 3.0} porta te podrškom za \textbf{4K} video rezolucijom.
\begin{figure}[h!]
\centering
\includegraphics[width=0.8\textwidth]{rpi-3-b-plus.jpg}
\captionsetup{justification=centering}
\caption{Raspberry Pi 3 B+}
\end{figure}
\begin{figure}[h!]
\centering
\includegraphics[width=0.8\textwidth]{rpi-4.png}
\captionsetup{justification=centering}
\caption{Raspberry Pi 4 B}
\end{figure}
\newline
\newline
Na laboratorijskim vježbama će se koristiti \textbf{Raspberry Bi 3 B}.
\newline
\newline
Tehničke karakteristike Raspberry Pi 3 B: \\
\textbf{SoC}: Broadcom BCM2837 \\
\textbf{CPU}: četverojezgreni ARM Cortex-A53, 1.2GHz \\
\textbf{GPU}: Broadcom VideoCore IV \\
\textbf{RAM}: 1GB LPDDR2 (900 MHz) \\
\textbf{Mreža}: 10/100 Ethernet, 2.4GHz 802.11n wireless \\
\textbf{Bluetooth}: Bluetooth 4.1 Classic, Bluetooth Low Energy \\
\textbf{Pohrana}: microSD \\
\textbf{GPIO}: 40-pin header, populated \\
\textbf{Priključci}: $4\times USB 2.0$, HDMI, 3.5mm analogni priključak,
 Ethernet, \textit{Camera Serial Interface} (CSI), \textit{Display Serial
 Interface} (DSI)
\begin{figure}%
\centering
\includegraphics[width=0.8\textwidth]{rpi-3-details.jpg}
\captionsetup{justification=centering}
\caption{Raspberry Pi 3 detalji}
\end{figure}
\begin{figure}[h!]
\centering
\includegraphics[width=0.8\textwidth]{rpi-3-pins.jpg}
\captionsetup{justification=centering}
\caption{Raspberry Pi 3 GPIO raspored}
\end{figure}

\clearpage
\section{Dodatna oprema}
Osim samog Raspberry Pi računala, potrebna je različita dodatna oprema ovisno
 o primjeni:

\begin{enumerate}
	\item napajanje 5V 2A, USB mikro priključak
	\item mikro SD memorijska kartica, kapaciteta min. 4GB klase A
	\item pretvornik USB na serijsku komunikaciju u obliku kabela (UART)
	\item pretvornik USB na Ethernet + Ethernet kabel
	\item dodatni uređaji za razvoj upravljačkih programa poput Nunchuk
		 upravljača.
\end{enumerate}

\begin{figure}%
    \centering
    \subfloat[Napajanje]{{\includegraphics[width=4cm]{rpi-charger.jpg}}}%
    \subfloat[SD kartica]{{\includegraphics[width=4cm]{rpi-sdcard.jpg}}}%
    \subfloat[USB UART]{{\includegraphics[width=4cm]{rpi-uart.jpg}}}%
    \qquad
    \subfloat[USB Ethernet]{{\includegraphics[width=4cm]{rpi-ethernet.jpg}}}%
    \subfloat[Nunchuk]{{\includegraphics[width=4cm]{rpi-nunchuk.jpg}}}%
    \subfloat[Kućište]{{\includegraphics[width=4cm]{rpi-case.jpg}}}%
    \caption{Raspberry Pi dodatna oprema}%
    \label{fig:oprema1}%
\end{figure}

\section{Osnovne postavke}
Ako na razvojnom računalu nemate vaš repozitorij \texttt{embedded\_linux},
 najprije ga klonirajte odgovarajućom naredbom s vašeg Gitlab računa u vaš
 \texttt{home} direktorij. Pozicionirajte se u direktorij
 \texttt{/home/rtrk/embedded\_linux}. Zatim povucite moguće promjene iz
 izvornog repozitorija pomoću naredbe:
\begin{lstlisting}[language=bash]
git remote add upstream
	https://gitlab.com/rgrbic/embedded_linux
git fetch upstream
git merge upstream/master
\end{lstlisting}
Instalirajte potrebne alate na razvojno računalo:
\begin{lstlisting}[language=bash]
sudo apt-get install gcc-arm-linux-gnueabihf
\end{lstlisting}

\section{Buildroot}
\textit{Build} sustav je skup alata koji omogućuje automatizaciju procesa
 izgradnje Linuxa za ugradbene računalne sustave. Preciznije rečeno,
 \textit{build} sustavi omogućuju izgradnju \textit{toolchain} alata,
 \textit{bootloadera}, \textit{kernela} i \textit{root} datotečnog sustava na
 temelju izvornog koda. Neki od poznatijih build sustava su Buidroot, Yocto,
 PTXdist i OpenEmbedded. \\
\newline
Za potrebe izgradnje Linuxa za raspoloživo Raspberry Pi računalo koristit će se
 \textit{Buildroot} alat. Otvorite terminal i preuzmite \textit{Buildroot}
 pomoću naredbe:
\begin{lstlisting}[language=bash]
git clone https://github.com/buildroot/buildroot
\end{lstlisting}
Nakon što se postupak kloniranja na vaše razvojno računalo završi, uđite u
 kreirani direktorij:
\begin{lstlisting}[language=bash]
cd buildroot
\end{lstlisting}
Primijetit ćete da ovaj direktorij ima nekoliko poddirektorija (koristite
 naredbu \texttt{ls}) od kojih su najvažniji (detalji se mogu pogledati na\\
 \url{https://buildroot.org/downloads/manual/manual.html}):\\
\newline
 \texttt{dl}/: sadrži arhive od upstream projekata koje je Buildroot izgradio\\
\newline
 \texttt{output}/: ovdje se nalaze svi među i završni rezultati prevođenja
 izvora\\
\newline
\texttt{host}/: ovo su različiti alati koje zahtijeva \textit{Buildroot}, a
 koji se izvršavaju na host računalu, uključujući izvršne datoteke
 toolchaina (\textit{output/host/usr/bin})\\
\newline
\texttt{images}/: ovo je najvažniji direktorij jer se u njemu nalaze rezultati
izgradnje (ovisno što je odabrano kod konfiguracije \textit{buildroota} –
 bootloader, kernel i jedan ili više root datotečnih sustava)\\
\newline
\textit{staging}/: ovo je simbolička veza prema \textit{sysrootu toolchaina}\\
\newline
\textit{target}/: ovo je gotovo cjeloviti \textit{root} datotečni sustav
 (međutim nije namijenjen za korištenje kao \textit{root} datotečni sustav na
 target računalu)\\
\newline
\textit{configs}/: ovdje se nalaze predefinirane konfiguracije za različite
 ploče (npr. \textit{raspberrypi3\_defconfig})\\
\newline
\textit{boards}/: ovdje se nalaze različite skripte koje definiraju način
 generiranja slike cijelog sustava (sdcard.img), post build procedure i sl.\\
\newline
Iako se za potrebe izgradnje Linuxa pomoću Buildroota može koristiti\\
 \texttt{raspberrypi3\_defconfig} koji se dodatno modificira prema željama
 korisnika naredbom \texttt{make menuconfig|nconfig|xconfig}, u vježbi će se
 iskoristiti postojeća konfiguracija. Kopirajte
 \texttt{raspberrypi3\_ferit\_defconfig} datoteku iz repozitorija
 \texttt{/home/rtrk/embedded\_linux/LV2/resources} u direktorij
 \texttt{/home/rtrk/buildroot/configs}. Nadalje, kopirajte datoteke\\
 \texttt{post-build.sh} i \texttt{genimage-raspberrypi3.cfg} u direktorij\\
 \texttt{/home/rtrk/buildroot/board/raspberrypi}.\\
\newline
Učitajte danu konfiguraciju i
 provjerite što je sve uključeno u nju pokretanjem sljedećih naredbi u
 direktoriju \texttt{/home/rtrk/buildroot}:
\begin{lstlisting}[language=bash]
make raspberrypi3_ferit_defconfig
make menuconfig
\end{lstlisting}
Pokrenite izgradnju pomoću naredbe:
\begin{lstlisting}[language=bash]
make
\end{lstlisting}

\end{document}
