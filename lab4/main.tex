\documentclass[11pt]{article}
\usepackage[croatian]{babel}  % Croatian typographical rules and hyphenation patterns
\usepackage[utf8]{inputenc}  	% Encoding of Croatian characters
\usepackage[T1]{fontenc}
\usepackage{ae,aecompl}     	% Type 1 fonts, similar to Computer Modern

\usepackage{microtype}				% Improves spacing

\usepackage{booktabs}					% Nice looking tables
\usepackage{enumerate}				% Additional options for listing of items in enumerate environment
\usepackage{algorithm2e}			% Writing pseudo-code
\usepackage{todonotes}				% Adding todo items
\usepackage{dirtree}					% Simple display of directory tree
\usepackage{hyperref}
\usepackage{graphicx}
\usepackage{subfig}
\usepackage{caption}
\usepackage{listings}
\usepackage{enumitem}
\usepackage{xcolor}

\graphicspath{ {./images/} }

\hypersetup{
    colorlinks=true,
    linkcolor=blue,
    filecolor=magenta,
    urlcolor=cyan,
}
\urlstyle{same}
\usepackage{graphics}
\title{
	\Large Sveučilište Josipa Jurja Strossmayera u Osijeku \\
	Fakultet elektrotehnike, računarstva i informacijskih tehnologija \\
	\vspace{4cm}
	\Large Kolegij: Linux u ugradbenim sustavima \\
	\vspace{4cm}
	\Large \textbf{Laboratorijska vježba 4:\\
	Nunchuk upravljački program}
	}
\date{}
\begin{document}
\maketitle
\thispagestyle{empty}
\newpage

\section{Uvod}
Cilj vježbe ja nastaviti izradu \texttt{Nunchuk} upravljačkog programa.
Proučavajući \texttt{Nunchuk} dokumentaciju pronaći način kako pročitati
stanja tipki \texttt{Nunchuk} uređaja. Stanja ispisati na standardnom izlazu.

\section{Postavke}
Za početak, postaviti varijable okruženja:
\begin{lstlisting}
export ARCH=arm
export CROSS_COMPILE=arm-linux-gnueabihf-
\end{lstlisting}
Nakon toga, pozicionirajte se direktorij
\texttt{/home/rtrk/embedded\_linux/LV4/solutions/nunchuk}. \texttt{Makefile}
ostaje isti kao i u prethodnoj vježbi. Datoteku \texttt{nunchuk.c} nadograđujemo
s prošle vježbe.

\section{Stanja \texttt{Nunchuk} uređaja}
Sljedeći korak u izradi upravljačkog programa \texttt{Nunchuk} uređaja je
čitanje stanja registara. Prvo moramo poslati uređaju komande za inicijalizaciju.
To je ujedno i zgodan način da bi se uvjerili da I\textsuperscript{2}C komunikacija
ispravno radi.\\
\newline
U funkciji \texttt{probe()} (pokreće se svaki puta kada je uređaj prepoznat)
treba implementirati sljedeće:
\begin{enumerate}
	\item Koristeći \texttt{i2c\_master\_send()} API uređaju poslati dva bajta:
		\texttt{0x40} i \texttt{0x00}. Provjeravajte povratnu vrijednost
		funkcije. U dokumentaciji na internetu pronađite primjere kako pravilno
		rukovati greškama koristeći koristeći istu funkciju.
	\item Nakon prošlog koraka sačekajte 1ms koristeći funkciju \texttt{udelay()}.
		Provjerite koja zaglavlja je potrebno uključiti za ovu funkciju.
	\item Na isti način pošaljite još i bajt \texttt{0x00}. Ovime je završena
		inicijalizacija uređaja.
\end{enumerate}
Prevedite modul i uvjerite se da nema grešaka prilikom prevođenja.

\section{Čitanje Nunchuk registra}
Napravite funkciju \texttt{nunchuk\_read\_registers()}. U ovoj funkciji:
\begin{enumerate}
	\item Počnite s pauzom 10ms pozivom funkcije \texttt{mdelay()}. To je
		trajanje neophodne pauze između dvije uzastopne I\textsuperscript{2}C
		operacije.
	\item Pošaljite bajt \texttt{0x00} na sabirnicu koristeći
		\texttt{i2c\_master\_send} API.
	\item Dodajte još jednu pauzu od 10ms.
	\item Pročitajte 6 bajtova s uređaja. Provjerite povratne vrijednosti.
\end{enumerate}

\section{Čitanje stanja Nunchuk tipki}
Vratite se na funkciju \texttt{probe()} gdje ćete funkciju
\texttt{nunchuk\_read\_registers()} pozvati dva puta (stanja na
I\textsuperscript{2}C sabirnici se mijenjaju tek nakon čitanja).
Nakon drugog poziva, odredite stanja tipki \textit{Z} i \textit{C} koja se
mogu pronaći u šestom bajtu kojeg ste pročitali.
\newline
\newline
Kao što je objašnjeno u dokumentu
\url{http://www.robotshop.com/media/files/PDF/inex-zx-nunchuck-datasheet.pdf}:
\begin{enumerate}
	\item Bit 0==0 označava da je tipka \textit{Z} pritisnuta.
	\item Bit 0==1 označava da tipka \textit{Z} nije pritisnuta.
	\item Bit 1==0 označava da je tipka \textit{C} pritisnuta.
	\item Bit 1==1 označava da tipka \textit{C} nije pritisnuta.
\end{enumerate}
Koristeći se logičkim operatorima napišite kod koji postavlja \texttt{zpressed}
cjelobrojnu varijablu kada je tipka \textit{Z} pritisnuta. Isto tako, napišite
kod koji postavlja \texttt{cpressed} cjelobrojnu varijablu kada je tipka
\textit{C} pritisnuta.
\newline
\newline
Na kraju funkcije \texttt{probe()} provjerite stanja novih varijabli i ispišite
koja je tipka pritisnuta na standardni izlaz.

\section{Testiranje}
Prevedite modul i učitajte ga. Uklonite modul. Nikakvi ispisi se ne bi trebali
pojaviti. Sada pritisnite tipku \textit{Z} ili \textit{C}. Ponovo učitajte i
uklonite modul:
\begin{lstlisting}[language=bash]
insmod /root/nunchuk/nunchuk.ko; rmmod nunchuk
\end{lstlisting}
Sada bi trebali vidjeti ispise s obzirom na pritisnute tipke. Ponovite istu
proceduru nekoliko puta s različitim kombinacijama tipki.
\newline
\newline
Na ovoj smo se laboratorijskoj vježbi učili kako učitati stanja uređaja na
I\textsuperscript{2}C sabirnici. Naravno, promjene smo mogli uočiti samo
učitavajući i uklanjajući modul, što u stvarnom životu baš i nije neko dobro
rješenje. Na sljedećoj ćemo laboratorijskoj vježbi raditi na mehanizmu kako
provjeravati stanja periodički.

\section{Spremanje rada}
Pozicionirajte se u direktorij \texttt{/home/rtrk/embedded\_linux/LV4/solutions/nunchuk}.
Izvršite \texttt{Git} komande:
\begin{lstlisting}[language=bash]
git add nunchuk.c
git add Makefile
git commit -m "laboratorijska vjezba 4 gotova"
git push origin master
\end{lstlisting}


\end{document}
