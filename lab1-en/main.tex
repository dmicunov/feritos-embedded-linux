\documentclass[11pt]{article}
\usepackage[english]{babel}  % Croatian typographical rules and hyphenation patterns
\usepackage{ae,aecompl}     	% Type 1 fonts, similar to Computer Modern

\usepackage{microtype}				% Improves spacing

\usepackage{booktabs}					% Nice looking tables
\usepackage{enumerate}				% Additional options for listing of items in enumerate environment
\usepackage{algorithm2e}			% Writing pseudo-code
\usepackage{todonotes}				% Adding todo items
\usepackage{dirtree}					% Simple display of directory tree
\usepackage{hyperref}
\hypersetup{
    colorlinks=true,
    linkcolor=blue,
    filecolor=magenta,
    urlcolor=cyan,
}
\urlstyle{same}
\usepackage{graphics}
\title{
	\Large Josip Juraj Strossmayer University of Osijek \\
	Faculty of Electrical Engineering, Computer Science and Information
	 Technologies\\
	\vspace{4cm}
	\Large Course: Linux in Embedded Systems \\
	\vspace{4cm}
	\Large \textbf{Laboratory exercise 1: Introduction to Linux environment}
	}
\date{}
\begin{document}
\maketitle
\thispagestyle{empty}
\newpage

\section{Introduction}
Exercise goals: Getting familiar with terminal. Working with package manager.
 Using Git version-control system. OS structure and file manipulation.
 Using a text editor. Compiling and running a program. Using network protocols.

\section{Booting operating system. Running a terminal}
After turning on computer, choose Ubuntu in boot menu. Username is
 \textbf{rtrk} and password is \textbf{rtrk}. Wait for the OS to boot.
 A graphical interface should appear on the screen. Get acquainted with the
 graphic interface. For example, try to find the Ethernet connection status
 of your computer.
 \newline
 \newline
 Pressing the \textit{Super} key (the key to the left of the \textit{Alt} key,
 the so-called \textit{Windows} key) opens \textit{Search} into which you then
 type in the terminal and press enter. It will launch the terminal where the
 command line is active, i.e. it is possible to type commands and press
 \textit{Enter} to execute the command entered. The commands are executed with
 the help of a \textit{Bash} shell. At the same time it is possible to have
 multiple running terminals (useful in many situations). After starting the
 terminal, the command shell is positioned in your home directory. Check this
 by typing the \textit{pwd} command.
\newline
\newline
\textbf{Note}: when writing commands, use the \textit{Tab} key to complete
commands, paths, filenames, etc. to avoid possible syntax errors.

\section{Using package manager}
Learn how to use the \textit{apt-get} package manager using the built-in
 help system (\textit{man apt-get}). Install git version-control system using
 the \textit{apt-get} tool. Set up your email address and username
using the git config command.

\section{Creating Gitlab account}
Register at \url{gitlab.com} (username form of \textit{namesurname} or
 \textit{xsurname} where {x} is the first letter of your name is recommended).
 If necessary, make a basic account configuration (e.g. create SSH keys). Make
 a fork of \textit{embedded\_linux} project from
 \url{https://gitlab.com/rgrbic/embedded\_linux.git}.
\newline
\newline
Clone the \textit{embedded\_linux} repository on your PC using the command
 \textit{git clone}. Create a new branch \textit{exercise\_work} using the
 \textit{git branch –b} command and checkout it immediately.

\section{Directory navigation and directory contents}
List the contents of the home directory with the \textit{ls} command. Using
 the \textit{man ls} command check all available options for this command. List
 the contents of the directory using the \textit{-l}, \textit{-a} options. How
 How to combine both options at the same time? What do these options provide
 when viewing directory contents? How are directories labeled
 and how files? What does \textit{.} represent in front of the name?
\end{document}
