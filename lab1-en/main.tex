\documentclass[11pt]{article}
\usepackage[english]{babel}  % Croatian typographical rules and hyphenation patterns
\usepackage{ae,aecompl}     	% Type 1 fonts, similar to Computer Modern

\usepackage{microtype}				% Improves spacing

\usepackage{booktabs}					% Nice looking tables
\usepackage{enumerate}				% Additional options for listing of items in enumerate environment
\usepackage{algorithm2e}			% Writing pseudo-code
\usepackage{todonotes}				% Adding todo items
\usepackage{dirtree}					% Simple display of directory tree
\usepackage{hyperref}
\hypersetup{
    colorlinks=true,
    linkcolor=blue,
    filecolor=magenta,
    urlcolor=cyan,
}
\urlstyle{same}
\usepackage{graphics}
\title{
	\Large Josip Juraj Strossmayer University of Osijek \\
	Faculty of Electrical Engineering, Computer Science and Information
	 Technologies\\
	\vspace{4cm}
	\Large Course: Linux in Embedded Systems \\
	\vspace{4cm}
	\Large \textbf{Laboratory exercise 1: Introduction to Linux environment}
	}
\date{}
\begin{document}
\maketitle
\thispagestyle{empty}
\newpage

\section{Introduction}
Exercise goals: Getting familiar with terminal. Working with package manager.
 Using Git version-control system. OS structure and file manipulation.
 Using a text editor. Compiling and running a program. Using network protocols.

\section{Booting operating system. Running a terminal}
After turning on computer, choose Ubuntu in boot menu. Username is
 \textbf{rtrk} and password is \textbf{rtrk}. Wait for the OS to boot.
 A graphical interface should appear on the screen. Get acquainted with the
 graphic interface. For example, try to find the Ethernet connection status
 of your computer.
 \newline
 \newline
 Pressing the \textit{Super} key (the key to the left of the \textit{Alt} key,
 the so-called \textit{Windows} key) opens \textit{Search} into which you then
 type in the terminal and press enter. It will launch the terminal where the
 command line is active, i.e. it is possible to type commands and press
 \textit{Enter} to execute the command entered. The commands are executed with
 the help of a \textit{Bash} shell. At the same time it is possible to have
 multiple running terminals (useful in many situations). After starting the
 terminal, the command shell is positioned in your home directory. Check this
 by typing the \textit{pwd} command.
\newline
\newline
\textbf{Note}: when writing commands, use the \textit{Tab} key to complete
commands, paths, filenames, etc. to avoid possible syntax errors.

\section{Using package manager}
Learn how to use the \textit{apt-get} package manager using the built-in
 help system (\textit{man apt-get}). Install git version-control system using
 the \textit{apt-get} tool. Set up your email address and username
using the git config command.

\section{Creating Gitlab account}
Register at \url{gitlab.com} (username form of \textit{namesurname} or
 \textit{xsurname} where {x} is the first letter of your name is recommended).
 If necessary, make a basic account configuration (e.g. create SSH keys). Make
 a fork of \textit{embedded\_linux} project from
 \url{https://gitlab.com/rgrbic/embedded\_linux.git}.
\newline
\newline
Clone the \textit{embedded\_linux} repository on your PC using the command
 \textit{git clone}. Create a new branch \textit{exercise\_work} using the
 \textit{git branch –b} command and checkout it immediately.

\section{Directory navigation and directory contents}
List the contents of the home directory with the \textit{ls} command. Using
 the \textit{man ls} command check all available options for this command. List
 the contents of the directory using the \textit{-l}, \textit{-a} options.
 How to combine both options at the same time? What do these options provide
 when viewing directory contents? How are directories labeled
 and how files? What does \textit{.} represent in front of the name?
 \newline
 \newline
Position yourself in the \textit{/home/rtrk/embedded\_linux/LV1/} resources
 directory. List only files or directories that begin with a lowercase letter.
 List all files ending in \textit{.txt}. List all files that do not finish
 with \textit{.txt}.
\newline
\newline
Using \textit{cd} command go to the root directory. Using which ways you can do
 that? Display the contents of the root directory. Study the resulting file
 system structure.
\newline
\newline
\textit{Linux} sees all devices as files. In which directory are the devices
 located? Display the detailed contents of that directory from your
 \textit{home} directory, using the absolute path. Only show devices/files that
 start with a small one the letter 'd'. Can you tell what the listed files
 represent?
\section{Manipulating files}
Go to directory \textit{/home/rtrk/embedded\_linux/LV1/resources}.
 Copy any files that contain the '9' character in their name to the \\
 \textit{/home/rtrk/embedded\_linux/LV1/resources/test9} directory. Delete all
 files in \textit{/home/rtrk/embedded\_linux/LV1/resources} that contain a
 character in their name '9'. Check the results by listing the contents of the
 directory.
\newline
\newline
Create a new directory in \textit{/home/rtrk/embedded\_linux/LV1/resources}
 called \textit{Test}. Copy all the files that have the name of the
 \textit{test\_datoteka} series into the \textit{Test} directory. Change the
 name of the \textit{Test} directory to \textit{test}. Delete the directory
 \textit{test}. Check the results by listing the contents of the directory.
\newline
\newline
Copy the \textit{/home/rtrk/embedded\_linux/LV1} directory to \\
\textit{/home/rtrk/embedded\_linux/LV1\_temp}. Check results by listing
 directory content. Delete the \textit{LV1\_temp} directory.
\newline
\newline
Delete any files that have the name of a series of \textit{test\_datoteka} in
 the directory \textit{/home/rtrk/embedded\_linux/lv1/resources}.
\newline
\newline
Unzip the \textit{/home/rtrk/embedded\_linux/LV1/resources/mbox.txt.bz2} file.
 Save the repository state by using \textit{Git} commands \textit{commit} and
 \textit{push}. When saving a state using \textit{commit} command write the
 appropriate message.
\section{File contents. Editing files. Permissions}
Display the contents of the \textit{mbox.txt} file using the \textit{cat},
 \textit{more}, and \textit{less} commands. Display the first 20 and last 20
 lines of this file. If necessary change the rights over the
 \textit{mbox.txt} file.
\newline
\newline
Display all email addresses that appear in the \textit{mbox-short.txt} file.
 Save all found email addresses in \textit{emails.txt} file. Change the
 permissions of \textit{emails.txt} file so the users can only read the file.
\newline
\newline
Find the email addresses in the \textit{mbox.txt} file and save them in
 \textit{emails1.txt}. Compare \textit{emails.txt} and \textit{emails1.txt}.
 Look for the numbers in each file in the directory
 \textit{/home/rtrk/embedded\_linux/LV1/resources/test9} and print the numbers.
 The result should be numbers 9, 19, 29, 39 etc. (each number is in a separate
 row). Write the results in the file \textit{numbers.txt} (without source file
 name). Display the contents of the file \textit{brojevi.txt}. Delete the
 \textit{test9} directory.
\newline
\newline
Start the \textit{nano} text editor. Open the \textit{emails.txt} file. Replace
 '@' character with a blank character.
\newline
\newline
Save the repository status using the \textit{commit} and \textit{push}
 \textit{git} commands. When saving a state using \textit{commit} command write
 the appropriate commit message.
\section{Compiling and running programs}
Compile \textit{cipher\_encryption.c} and \textit{cipher\_decryption.c} using
 \textit{gcc} and generate encrypt and decrypt executables. Encrypt the message
 "Introduction to Linux" using the generated encrypt executable file. Use the
 key '8'. Decrypt the message using the decrypt executable file. Modify the
 original code of \textit{cipher\_encryption.c} to save the results to a file
 named \textit{message.txt} and modify \textit{cipher\_decryption.c} to
 decrypt text from \textit{message.txt}. Translate the modified source code.
\newline
\newline
Try translating the source code with \textit{-Wall} enabled. Why the \textit{C}
 function \textit{gets} is not safe? How can you solve this problem?
\newline
\newline
 For all generated files as well as the source code, change the permissions so
that they cannot be modified, read or executed by other users.
\newline
\newline
Save the repository status to your local database as well as to a remote
 computer using the \textit{Git} \textit{commit} and \textit{push} commands.
 When saving a state using the \textit{commit} command write the appropriate
 message.
\section{Fetching files with \textit{wget}. \textit{SSH} communication}
Download the HTML file from the URL
 \url{https://www.ferit.unios.hr/studiji/sveucilisni-diplomski-studij} using
 \textit{wget}. Use the grep command to display all rows in a file that
 do not contain word "programiranje". If the command does not return any
 result, check with \textit{man grep} how to read binaries as if they were
 text.
\newline
\newline
The next exercise is done in pairs. An \textit{SSH} connection is required
 between two computers. An \textit{opensh-server} must be installed on one
 computer. You need to create a new user on the same computer. From another
 computer connect using the \textit{SSH} connection to a computer running the
\textit{SSH server} using the data for the created user. By running the
\textit{encrpyt} program which located in
\textit{/home/rtrk/embedded\_linux/LV1/resources/} leave encrypted message
 to a \textit{rtrk} user in file \textit{message.txt}. Exchange the encryption
key orally. The \textit{rtrk} user on the server needs to decrypt the message.
\section{Saving work}
Return to the git main branch using the \textit{git checkout master} command.
 Save all the commands you entered while working on the exercise to the file
 using command:
\newline
\newline
\textit{history} > <your\_surname.txt> (e.g. \textit{history} > john\_smith.txt)
\newline
\newline
Transfer the file to \textit{/home/rtrk/embedded\_linux/LV1/solutions/}. Save
 changes to the local repository using the \textit{git commit} command as well
 as yours remote account with \textit{git push -u origin master}. Create merge
 request on the Gitlab page to add your file to the original repository
 \textit{embedded\_linux}.
\section{Useful commands}
The following commands are useful: \textit{ls, grep, pwd, cd, mkdir, rm, cp,
 apt-get, diff, git push, git commit, git checkout, git pull, cat, more, less,
 nano, chown}.
\end{document}
