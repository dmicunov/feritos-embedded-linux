\documentclass[11pt]{article}
\usepackage[english]{babel}  % Croatian typographical rules and hyphenation patterns
\usepackage{ae,aecompl}     	% Type 1 fonts, similar to Computer Modern

\usepackage{microtype}				% Improves spacing

\usepackage{booktabs}					% Nice looking tables
\usepackage{enumerate}				% Additional options for listing of items in enumerate environment
\usepackage{algorithm2e}			% Writing pseudo-code
\usepackage{todonotes}				% Adding todo items
\usepackage{dirtree}					% Simple display of directory tree
\usepackage{hyperref}

\usepackage{graphicx}
\usepackage{subfig}
\usepackage{caption}
\usepackage{listings}

\hypersetup{
    colorlinks=true,
    linkcolor=blue,
    filecolor=magenta,
    urlcolor=cyan,
}
\urlstyle{same}
\usepackage{graphics}

\graphicspath{ {./images/} }

\title{
	\Large Josip Juraj Strossmayer University of Osijek \\
	Faculty of Electrical Engineering, Computer Science and Information
	 Technologies\\
	\vspace{4cm}
	\Large Course: Linux in Embedded Systems \\
	\vspace{4cm}
	\Large \textbf{Laboratory exercise 3: Nunchuk device driver}
	}
\date{}
\begin{document}
\maketitle
\thispagestyle{empty}
\newpage

\section{Introduction}
The goal of the exercise is to continue making the Nunchuk driver.
Check the Nunchuk documentation to find a way to read the status of the Nunchuk
device keys. Print states using the standard output.

\section{Setup}
Set environment variables:
\begin{lstlisting}
export ARCH=arm
export CROSS_COMPILE=arm-linux-gnueabihf-
\end{lstlisting}
After that, position yourself in the
\texttt{/home/rtrk/embedded\_linux/LV4/solutions/nunchuk} directory. The
\texttt{Makefile} remains the same as in the previous exercise. We are upgrading
the \texttt{nunchuk.c} file from the previous exercise.

\section{States of the \texttt{Nunchuk} device}
The next step in creating a Nunchuk device driver is to read the status of the
registers. We must first send to the device an initialization command. It is
also a convenient way to make sure that I\textsuperscript{2}C communication is
working properly.
\newline
\newline
In the \texttt{probe()} (starts each time the device is recognized) the
following should be implemented:
\begin{enumerate}
	\item Using the \texttt{i2c\_master\_send()} API, send two bytes to the
		device: 0x40 and 0x00. Check the return value of the function. See
		online documentation for examples of how to properly handle errors using
		the same function.
	\item After the last step, wait 1ms using the \texttt{udelay()} function.
		Check which headers to include for this feature.
	\item Also send byte 0x00 in the same way. This completes the initialization
		of the device.
Compile the module and make sure there are no compilation errors.
\end{enumerate}

\section{Reading the Nunchuk register}
Create the \texttt{nunchuk\_read\_registers()} function. In this function:
\begin{enumerate}
	\item Start with a 10ms pause by calling \texttt{mdelay()}. This is the
		duration of the necessary pause between two consecutive
		I\textsuperscript{2}C operations.
	\item Send the 0x00 byte to the bus using the \texttt{i2c\_master\_send}
		API.
	\item Add another 10ms pause.
	\item Read 6 bytes from the device. Check the return values.
\end{enumerate}

\section{Reading the status of Nunchuk keys}
Return to the \texttt{probe()} where you will call the
\texttt{nunchuk\_read\_registers()} function twice (states on the
I\textsuperscript{2}C bus change only after reading).
After the second call, determine the states of the \textit{Z} and
\textit{C} keys that can be found in the sixth byte that you read.
\newline
\newline
As explained in the document
\url{http://www.robotshop.com/media/files/PDF/inex-zx-nunchuck-datasheet.pdf}:
\begin{enumerate}
	\item Bit 0 == 0 indicates that the \textit{Z} key is pressed.
	\item Bit 0 == 1 indicates that the \textit{Z} key is not pressed.
	\item Bit 1 == 0 indicates that the \textit{C} key is pressed.
	\item Bit 1 == 1 indicates that the \textit{C} key is not pressed.
\end{enumerate}
Using logical operators, write code that sets the \texttt{zpressed} integer
variable when the \textit{Z} key is pressed. Also, write the code that sets the
\texttt{cpressed} integer variable when the \textit{C} key is pressed.
\newline
\newline
At the end of the \texttt{probe()} function, check the states of the new
variables and print which key is pressed to standard output.

\section{Testing}
Translate the module and load it. Remove the module. No printouts should appear.
Now press the Z or C. Insert and remove module:
\begin{lstlisting}[language=bash]
insmod /root/nunchuk/nunchuk.ko; rmmod nunchuk
\end{lstlisting}
You should now see the prints with the keys pressed. Repeat the same procedure
several times with different key combinations.
\newline
\newline
In this lab we learned how to load device states on an I\textsuperscript{2}C
bus. Of course, we could only see the changes by loading and removing the
module, which is not really a good solution in real life. In the next lab, we
will work on a mechanism for checking states periodically.

\section{Saving work}
Position yourself in the
\texttt{/home/rtrk/embedded\_linux/LV4/solutions/nunchuk} directory.
Execute \texttt{Git} commands:
\begin{lstlisting}[language=bash]
git add nunchuk.c
git add Makefile
git commit -m "laboratory exercise 4 done"
git push origin master
\end{lstlisting}
\end{document}
