\documentclass[11pt]{article}
\usepackage[english]{babel}  % Croatian typographical rules and hyphenation patterns
\usepackage{ae,aecompl}     	% Type 1 fonts, similar to Computer Modern

\usepackage{microtype}				% Improves spacing

\usepackage{booktabs}					% Nice looking tables
\usepackage{enumerate}				% Additional options for listing of items in enumerate environment
\usepackage{algorithm2e}			% Writing pseudo-code
\usepackage{todonotes}				% Adding todo items
\usepackage{dirtree}					% Simple display of directory tree
\usepackage{hyperref}

\usepackage{graphicx}
\usepackage{subfig}
\usepackage{caption}
\usepackage{listings}

\hypersetup{
    colorlinks=true,
    linkcolor=blue,
    filecolor=magenta,
    urlcolor=cyan,
}
\urlstyle{same}
\usepackage{graphics}

\graphicspath{ {./images/} }

\title{
	\Large Josip Juraj Strossmayer University of Osijek \\
	Faculty of Electrical Engineering, Computer Science and Information
	 Technologies\\
	\vspace{4cm}
	\Large Course: Linux in Embedded Systems \\
	\vspace{4cm}
	\Large \textbf{Laboratory exercise 5: Nunchuk device driver\\
					and polling mechanism}
	}
\date{}
\begin{document}
\maketitle
\thispagestyle{empty}
\newpage

\section{Introduction}
The goal of this exercise is to allow the user space part of the Linux
operating system to access the events of an external device using the Linux
kernel polling mechanism. Complete support for the functionality of the entire
Nunchuk device.

\section{Setup}
First, set environment variables:
\begin{lstlisting}
export ARCH=arm
export CROSS_COMPILE=arm-linux-gnueabihf-
\end{lstlisting}
After that, position yourself in the \texttt{/home/profesor/embedded\_linux/LV5/solutions/nunchuk}
directory.
The \texttt{Makefile} remains the same as in the previous exercise. We are
upgrading the \texttt{nunchuk.c} file from the previous exercise.

\section{Enabling polling in Linux kernel}
\texttt{Nunchuk} does not have any interrupts to notify the Linux kernel that
 it's status has changed. Therefore, one way to detect change is by using a
 polling mechanism.
\newline
\newline
Linux kernel needs to be compiled again. Position yourself directly
where the Linux kernel source code should be in the path
 \texttt{/home/profesor/buildroot/output/build/linux-custom}. Open the
 \texttt{menuconfig} menu using the command:
\begin{lstlisting}
make menuconfig
\end{lstlisting}
If an error occurred while running the command, the \texttt{libncurses} library
 on the development computer was probably not installed. Install the library
 using the command:
\begin{lstlisting}
sudo apt-get install libncurses-dev
\end{lstlisting}
Once you have entered the \texttt{menuconfig} menu, you need to search for the
\texttt{INPUT\_POLLDEV} and \texttt{INPUT\_EVDEV} configuration items and enable
them. Use the slash (/) key to search the settings. Press the '/' key and search
for the '\texttt{INPUT\_POLLDEV}' setting. Numerically select (e.g. '1') to
 select a setting. Set it to “yes” by using the space bar on the keyboard. Do
 the same for the \texttt{INPUT\_EVDEV} setting. Save the new configuration and
 exit the menu. Recompile the Linux kernel and copy the new zImage file to the
 \texttt{/var/lib/tftpboot} directory. Restart the board.

\end{document}
