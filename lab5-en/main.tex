\documentclass[11pt]{article}
\usepackage[english]{babel}  % Croatian typographical rules and hyphenation patterns
\usepackage{ae,aecompl}     	% Type 1 fonts, similar to Computer Modern

\usepackage{microtype}				% Improves spacing

\usepackage{booktabs}					% Nice looking tables
\usepackage{enumerate}				% Additional options for listing of items in enumerate environment
\usepackage{algorithm2e}			% Writing pseudo-code
\usepackage{todonotes}				% Adding todo items
\usepackage{dirtree}					% Simple display of directory tree
\usepackage{hyperref}

\usepackage{graphicx}
\usepackage{subfig}
\usepackage{caption}
\usepackage{listings}

\hypersetup{
    colorlinks=true,
    linkcolor=blue,
    filecolor=magenta,
    urlcolor=cyan,
}
\urlstyle{same}
\usepackage{graphics}

\graphicspath{ {./images/} }

\title{
	\Large Josip Juraj Strossmayer University of Osijek \\
	Faculty of Electrical Engineering, Computer Science and Information
	 Technologies\\
	\vspace{4cm}
	\Large Course: Linux in Embedded Systems \\
	\vspace{4cm}
	\Large \textbf{Laboratory exercise 5: Nunchuk device driver\\
					and polling mechanism}
	}
\date{}
\begin{document}
\maketitle
\thispagestyle{empty}
\newpage

\section{Introduction}
The goal of this exercise is to allow the user space part of the Linux
operating system to access the events of an external device using the Linux
kernel polling mechanism. Complete support for the functionality of the entire
Nunchuk device.

\section{Setup}
First, set environment variables:
\begin{lstlisting}
export ARCH=arm
export CROSS_COMPILE=arm-linux-gnueabihf-
\end{lstlisting}
After that, position yourself in the \texttt{/home/profesor/embedded\_linux/LV5/solutions/nunchuk}
directory.
The \texttt{Makefile} remains the same as in the previous exercise. We are
upgrading the \texttt{nunchuk.c} file from the previous exercise.

\section{Enabling polling in Linux kernel}
\texttt{Nunchuk} does not have any interrupts to notify the Linux kernel that
 it's status has changed. Therefore, one way to detect change is by using a
 polling mechanism.
\newline
\newline
Linux kernel needs to be compiled again. Position yourself directly
where the Linux kernel source code should be in the path
 \texttt{/home/profesor/buildroot/output/build/linux-custom}. Open the
 \texttt{menuconfig} menu using the command:
\begin{lstlisting}
make menuconfig
\end{lstlisting}
If an error occurred while running the command, the \texttt{libncurses} library
 on the development computer was probably not installed. Install the library
 using the command:
\begin{lstlisting}
sudo apt-get install libncurses-dev
\end{lstlisting}
Once you have entered the \texttt{menuconfig} menu, you need to search for the
\texttt{INPUT\_POLLDEV} and \texttt{INPUT\_EVDEV} configuration items and enable
them. Use the slash (/) key to search the settings. Press the '/' key and search
for the '\texttt{INPUT\_POLLDEV}' setting. Numerically select (e.g. '1') to
 select a setting. Set it to “yes” by using the space bar on the keyboard. Do
 the same for the \texttt{INPUT\_EVDEV} setting. Save the new configuration and
 exit the menu. Recompile the Linux kernel and copy the new \texttt{zImage}
 file to the \texttt{/var/lib/tftpboot} directory. Restart the board.

\section{Upgrading \texttt{Nunchuk} device driver}
The input device must first be added to the system. Make the following changes:
\begin{enumerate}
	\item Declare a pointer to the \texttt{input\_polled\_dev} structure in
		function \texttt{probe()}. You can call it \texttt{polled\_input}.
	\item Allocate a new structure in the same function using the\\
		\texttt{input\_allocate\_polled\_device()} function.
	\item Define a pointer to the \texttt{input\_dev} structure. You can call it
		\texttt{input}. Allocation is not required because it is already part of
		\texttt{input\_polled\_dev} structure and is simultaneously allocated.
		We will only use it as a shortcut to keep the code as simple as
		possible.
	\item In function \texttt{probe()}, add the input device to the system by
		calling \texttt{input\_register\_polled\_device()}.
\end{enumerate}
Make sure the module compiles successfully. Add missing headers as needed.

\section{Error handling in \texttt{probe()} function}
In the code you wrote, make sure you properly handle errors:
\begin{enumerate}
	\item Always check the return values of the functions and log the causes of
		errors.
	\item If the call to the \texttt{input\_register\_polled\_device()} function
		fails, you must release the \texttt{input\_polled\_dev} structure before
		returning error code, that is, before the function is completed. If
		not doing this will cause these memory leaks in the kernel. Generally
		speaking, failing to let go of what is previously allocated or
		registered can lead to problems, that is, it can prevent the module from
		reloading.
\end{enumerate}
It is recommended to use the \texttt{goto} command for correct and nice error
 management without unnecessary and ugly copying of the code.

\section{Implementing function \texttt{remove()}}
All resources occupied or registered by the \texttt{probe()} function must be
 released in this function.
\newline
Before we get started with the \texttt{remove()} function, let's add the
 following structure definition at the beginning of the file:
\begin{lstlisting}
struct nunchuk_dev {
	struct input_polled_dev *polled_input;
	struct i2c_client *i2c_client;
};
\end{lstlisting}
Then allocate the structure for the new device:
\begin{lstlisting}
nunchuk = devm_kzalloc(&client->dev, sizeof(struct nunchuk_dev),
	GFP_KERNEL);
if (!nunchuk) {
	dev_err(&client->dev, "Failed to allocate memory\n");
	return -ENOMEM;
}
\end{lstlisting}
Notice that \texttt{devm\_kzalloc} is a function to allocate memory in the
 kernel.
Furthermore, in the \texttt{probe()} function, add:
\begin{lstlisting}
nunchuk->i2c_client = client;
nunchuk->polled_input = polled_input;
polled_input->private = nunchuk;
i2c_set_clientdata(client, nunchuk);
input = polled_input->input;
input->dev.parent = &client->dev;
\end{lstlisting}
Now we have everything we need to implement the \texttt{remove()} function:
\begin{lstlisting}
static int
nunchuk_remove(struct i2c_client *client)
{
	struct nunchuk_dev *dev;
	dev = i2c_get_clientdata(client);
	input_unregister_polled_device(dev->polled_input);
	input_free_polled_device(dev->polled_input);
	devm_kfree(&client->dev, dev);

	return 0;
}
\end{lstlisting}
Find similarities to calls within the \texttt{probe()} function.
\newline
Recompile the module, load it and remove it several times to make sure
 everything is logged in successfully or logged out.

\section{Adding correct input device registration information}
We need to add more information before registering the entry structure. This is
 the reason why you are currently receiving alerts when loading a module:
\begin{lstlisting}[language=bash]
input: Unspecified device as /devices/virtual/input/input0
\end{lstlisting}
Add the following lines of code:
\begin{lstlisting}[language=bash]
input->name = "Wii Nunchuk";
input->id.bustype = BUS_I2C;
set_bit(EV_ABS, input->evbit);
set_bit(ABS_X, input->absbit);
set_bit(ABS_Y, input->absbit);
set_bit(ABS_RX, input->absbit);
set_bit(ABS_RY, input->absbit);
set_bit(ABS_RZ, input->absbit);
set_bit(EV_KEY, input->evbit);
set_bit(BTN_C, input->keybit);
set_bit(BTN_Z, input->keybit);
set_bit(INPUT_PROP_ACCELEROMETER, input->propbit);
\end{lstlisting}
Also, add function calls that eliminate accelerometer noise and joystick noise:
\begin{lstlisting}[language=bash]
input_set_abs_params(input, ABS_X, 30, 220, 4, 8);
if (!input->absinfo) {
dev_err(&client->dev, "input->absinfo allocation failed\n");
	goto exit_nunchuk_dev;
}
input_set_abs_params(input, ABS_Y, 40, 200, 4, 8);
input_set_abs_params(input, ABS_RX, 0, 0x3ff, 4, 8);
input_set_abs_params(input, ABS_RY, 0, 0x3ff, 4, 8);
input_set_abs_params(input, ABS_RZ, 0, 0x3ff, 4, 8);
\end{lstlisting}
Declare a new \texttt{polling} function and set an interval of 50ms:
\begin{lstlisting}[language=bash]
polled_input->poll = nunchuk_poll;
polled_input->poll_interval = 50;
\end{lstlisting}
The role of the \texttt{nunchuk\_poll} function is to calculate all states of
 the \texttt{Nunchuk} device and to submit the event states:
\begin{lstlisting}
input_event(polled_input->input, EV_KEY, BTN_Z, z_pressed);
input_event(polled_input->input, EV_KEY, BTN_C, c_pressed);
input_report_abs(polled_input->input, ABS_RX, accelerometer[0]);
input_report_abs(polled_input->input, ABS_RY, accelerometer[1]);
input_report_abs(polled_input->input, ABS_RZ, accelerometer[2]);
input_report_abs(polled_input->input, ABS_X, joystick[0]);
input_report_abs(polled_input->input, ABS_Y, joystick[1]);
input_sync(polled_input->input);
\end{lstlisting}

\section{Module testing}
Testing an input device is easy when using an \texttt{evtest} application that
 is included in the \textit{root filesystem}. Run the \texttt{evtest}
 application as follows:
\begin{lstlisting}
evtest /dev/input/event0
\end{lstlisting}
Press different key combinations to see how specific events are logged through
 the \texttt{evtest} application.

\section{Saving work}
Position yourself in the
\texttt{/home/profesor/embedded\_linux/LV5/solutions/nunchuk} directory.
Execute \texttt{Git} commands:
\begin{lstlisting}[language=bash]
git add nunchuk.c
git add Makefile
git commit -m "laboratory exercise 5 done"
git push origin master
\end{lstlisting}
\end{document}
\end{document}
