\documentclass[11pt]{article}
\usepackage[croatian]{babel}  % Croatian typographical rules and hyphenation patterns
\usepackage[utf8]{inputenc}  	% Encoding of Croatian characters
\usepackage[T1]{fontenc}
\usepackage{ae,aecompl}     	% Type 1 fonts, similar to Computer Modern

\usepackage{microtype}				% Improves spacing

\usepackage{booktabs}					% Nice looking tables
\usepackage{enumerate}				% Additional options for listing of items in enumerate environment
\usepackage{algorithm2e}			% Writing pseudo-code
\usepackage{todonotes}				% Adding todo items
\usepackage{dirtree}					% Simple display of directory tree
\usepackage{hyperref}
\usepackage{graphicx}
\usepackage{subfig}
\usepackage{caption}
\usepackage{listings}
\usepackage{enumitem}
\usepackage{xcolor}

\graphicspath{ {./images/} }

\hypersetup{
    colorlinks=true,
    linkcolor=blue,
    filecolor=magenta,
    urlcolor=cyan,
}
\urlstyle{same}
\usepackage{graphics}
\title{
	\Large Sveučilište Josipa Jurja Strossmayera u Osijeku \\
	Fakultet elektrotehnike, računarstva i informacijskih tehnologija \\
	\vspace{4cm}
	\Large Kolegij: Linux u ugradbenim sustavima \\
	\vspace{4cm}
	\Large \textbf{Laboratorijska vježba 5:\\
	Nunchuk upravljački program i\\
	\textit{polling} mehanizam}
	}
\date{}
\begin{document}
\maketitle
\thispagestyle{empty}
\newpage

\section{Uvod}
Cilj ove vježbe je omogućiti \textit{user space} dijelu \textit{Linux}
operativnog sustava pristup događajima vanjskog uređaja (\textit{kernel space})
korištenjem \textit{polling} mehanizma \textit{Linux} jezgre. Dovršiti
podršku za funkcionalnost cijelog \textit{Nunchuk} uređaja.

\section{Postavke}
Za početak, postaviti varijable okruženja:
\begin{lstlisting}
export ARCH=arm
export CROSS_COMPILE=arm-linux-gnueabihf-
\end{lstlisting}
Nakon toga, pozicionirajte se direktorij
\texttt{/home/rtrk/embedded\_linux/LV5/solutions/nunchuk}. \texttt{Makefile}
ostaje isti kao i u prethodnoj vježbi. Datoteku \texttt{nunchuk.c} nadograđujemo
s prošle vježbe.

\section{Dodavanje podrške za \textit{polling} mehanizam u \textit{Linux} jezgru}
\textit{Nunchuk} nema prekide (engl. \textit{interrupt}) s kojima bi obavijestio
\textit{Linux} jezgru da je njegovo stanje promijenjeno. Stoga je jedan način
otkrivanja promjena korištenjem \textit{polling} mehanizma.
\newline
\newline
Potrebno je ponovno prevesti \textit{Linux} jezgru. Pozicionirajte se u
direktorij gdje se nalazi izvorni kod \textit{Linux} jezgre koji bi trebao biti
na putanji \texttt{/home/rtrk/buildroot/output/build/linux-custom}.
Otvorite \texttt{menuconfig} izbornik korištenjem naredbe:
\begin{lstlisting}
make menuconfig
\end{lstlisting}
Ako se prilikom pokretanja naredbe javila greška, vjerojatno nije instalirana
biblioteka \texttt{libncurses} na razvojnom računalu.\\
Instalirajte biblioteku korištenjem naredbe:
\begin{lstlisting}
sudo apt-get install libncurses-dev
\end{lstlisting}
Kada ste ušli u \texttt{menuconfig} izbornik, potrebno je potražiti
konfiguracijske stavke \texttt{INPUT\_POLLDEV} i \texttt{INPUT\_EVDEV} i
uključiti ih. S tipkom \textit{slash} (\texttt{/}) pretražujete postavke.
Pritisnite tipku '/' te pretražite postavku "\texttt{INPUT\_POLLDEV}".
Numeričkim izborom (npr. '1') izaberete postavku. Postavite ju na "\texttt{yes}"
korištenjem tipke razmaka na tipkovnici. Isto napravite i za postavku
\texttt{INPUT\_EVDEV}. Spremite novu konfiguraciju i izađite iz izbornika.
Ponovno prevedite \texttt{Linux} jezgru te kopirajte novu \texttt{zImage}
datoteku u direktorij \texttt{/var/lib/tftpboot}. Ponovno pokrenite ploču.

\section{Nadogradnja \textit{Nunchuk} upravljačkog programa}
Prvo treba dodati ulazni uređaj u sustav. Napravite sljedeće izmjene:
\begin{enumerate}
	\item Deklarirajte pokazivač na \texttt{input\_polled\_dev} strukturu u
		funkciji \texttt{probe()}. Možete ga nazvati \texttt{polled\_input}.
	\item Alocirajte novu strukturu u istoj funkciji koristeći funkciju
		\texttt{input\_allocate\_polled\_device()}.
	\item Definirajte pokazivač na \texttt{input\_dev} strukturu. Možete ga
		nazvati input. Alokacija nije potrebna jer je on već dio
		\texttt{input\_polled\_dev} strukture i istovremeno je alociran.
		Koristit ćemo ga samo kao prečicu da bismo kod održali što
		jednostavnijim.
	\item U funkciji \texttt{probe()} dodajte ulazni uređaj u sistem pozivom
		\texttt{input\_register\_polled\_device()}.
\end{enumerate}
Uvjerite se da se modul uspješno prevodi. Dodajte po potrebi zaglavlja koja
nedostaju.

\section{Rukovanje greškama u \texttt{probe()} funkciji}
U kodu kojeg ste napisali, provjerite da li na odgovarajući način upravljate
greškama:
\begin{enumerate}
	\item Uvijek na odgovarajući način provjerite povratne vrijednosti
		funkcija i evidentirajte uzroke grešaka.
	\item Ako je poziv funkcije \texttt{input\_register\_polled\_device()}
		neuspješan, morate osloboditi \texttt{input\_polled\_dev} strukturu
		prije nego što vratite kod greške, odnosno prije nego što se završi
		izvršavanje funkcije. Ako to ne uradite prouzročit će te curenje
		memorije (engl. \textit{memory leak}) u kernelu. Općenito govoreći,
		neuspješno otpuštanje onoga što je prethodno bilo alocirano ili
		registrirano može dovesti do problema, odnosno može spriječiti ponovno
		učitavanje modula.
\end{enumerate}

Za korektno i lijepo upravljanje greškama bez suvišnog i ružnog kopiranja koda
preporuča se korištenje naredbe \texttt{goto}.

\section{Implementacija funkcije \texttt{remove()}}
U ovoj funkciji je potrebno otpustiti sve resurse koje je zauzela ili
registrirala funkcija \texttt{probe()}.
\newline
\newline
Prije nego što krenemo se implementacijom funkcije \texttt{remove()},
dodajte sljedeću definiciju strukture na počektu datoteke:
\begin{lstlisting}
struct nunchuk_dev {
	struct input_polled_dev *polled_input;
	struct i2c_client *i2c_client;
};
\end{lstlisting}
Zatim, u funkciji \texttt{probe()} deklarirajte instancu ove strukture:
\begin{lstlisting}
struct nunchuk_dev *nunchuk;
\end{lstlisting}
Zatim alocirajte strukturu za novi uređaj:
\begin{lstlisting}
nunchuk = devm_kzalloc(&client->dev, sizeof(struct nunchuk_dev),
	GFP_KERNEL);
if (!nunchuk) {
	dev_err(&client->dev, "Failed to allocate memory\n");
	return -ENOMEM;
}
\end{lstlisting}
Primijetite da je \texttt{devm\_kzalloc} funkcija za alociranje memorije u
kernelu.
Nadalje, u funkciji \texttt{probe()} dodajte:
\begin{lstlisting}
nunchuk->i2c_client = client;
nunchuk->polled_input = polled_input;
polled_input->private = nunchuk;
i2c_set_clientdata(client, nunchuk);
input = polled_input->input;
input->dev.parent = &client->dev;
\end{lstlisting}
Sada imamo sve što je potrebno za implementiranje funkcije \texttt{remove()}:
\begin{lstlisting}
static int
nunchuk_remove(struct i2c_client *client)
{
	struct nunchuk_dev *dev;
	dev = i2c_get_clientdata(client);
	input_unregister_polled_device(dev->polled_input);
	input_free_polled_device(dev->polled_input);
	devm_kfree(&client->dev, dev);

	return 0;
}
\end{lstlisting}
Pronađite analogije s pozivima unutar funkcije \texttt{probe()}.
\newline
\newline
Prevedite ponovno modul, učitajte ga te ga uklonite nekoliko puta kako bi ste
se uvjerili da je sve uspješno prijavljeno, odnosno odjavljeno.

\section{Dodavanje ispravnih informacija o registraciji ulaznog uređaja}
Prije registracije ulazne strukture trebamo dodati još informacija.
To je razlog zbog kojeg u ovom trenutku dobivate upozorenja
prilikom učitavanja modula, odnosno prijave novog uređaja:
\begin{lstlisting}[language=bash]
input: Unspecified device as /devices/virtual/input/input0
\end{lstlisting}
Dodajte sljedeće linije koda:
\begin{lstlisting}[language=bash]
input->name = "Wii Nunchuk";
input->id.bustype = BUS_I2C;
set_bit(EV_ABS, input->evbit);
set_bit(ABS_X, input->absbit);
set_bit(ABS_Y, input->absbit);
set_bit(ABS_RX, input->absbit);
set_bit(ABS_RY, input->absbit);
set_bit(ABS_RZ, input->absbit);
set_bit(EV_KEY, input->evbit);
set_bit(BTN_C, input->keybit);
set_bit(BTN_Z, input->keybit);
set_bit(INPUT_PROP_ACCELEROMETER, input->propbit);
\end{lstlisting}
Također, dodajte pozive funkcija koje otklanjaju šum na akcelerometru te
komandnoj ručici:
\begin{lstlisting}[language=bash]
input_set_abs_params(input, ABS_X, 30, 220, 4, 8);
if (!input->absinfo) {
	dev_err(&client->dev, "input->absinfo allocation failed\n");
	goto exit_nunchuk_dev;
}
input_set_abs_params(input, ABS_Y, 40, 200, 4, 8);
input_set_abs_params(input, ABS_RX, 0, 0x3ff, 4, 8);
input_set_abs_params(input, ABS_RY, 0, 0x3ff, 4, 8);
input_set_abs_params(input, ABS_RZ, 0, 0x3ff, 4, 8);
\end{lstlisting}
Deklarirajte novu \texttt{polling} funkciju i postavite interval od 50ms:
\begin{lstlisting}[language=bash]
polled_input->poll = nunchuk_poll;
polled_input->poll_interval = 50;
\end{lstlisting}
Uloga \texttt{nunchuk\_poll} funkcije je izračunati sva stanja \textit{Nunchuk}
uređaja te predati stanja u obliku događaja (engl. \textit{event}):
\begin{lstlisting}
input_event(polled_input->input, EV_KEY, BTN_Z, z_pressed);
input_event(polled_input->input, EV_KEY, BTN_C, c_pressed);
input_report_abs(polled_input->input, ABS_RX, accelerometer[0]);
input_report_abs(polled_input->input, ABS_RY, accelerometer[1]);
input_report_abs(polled_input->input, ABS_RZ, accelerometer[2]);
input_report_abs(polled_input->input, ABS_X, joystick[0]);
input_report_abs(polled_input->input, ABS_Y, joystick[1]);
input_sync(polled_input->input);
\end{lstlisting}
\section{Testiranje modula}
Testiranje ulaznog uređaja je jednostavno kada se koristi \textit{evtest}
aplikacija koja je uključena i u \textit{root filesystem}.
Pokrenite \textit{evtest} aplikaciju na način:
\begin{lstlisting}
evtest /dev/input/event0
\end{lstlisting}
Pritisnite različite kombinacije tipaka i uočite kako su određeni događaji
prijavljeni kroz \textit{evtest} aplikaciju.

\section{Spremanje rada}
Pozicionirajte se u direktorij \texttt{/home/profesor/embedded\_linux/LV5/solutions/nunchuk}.
Izvršite \texttt{Git} komande:
\begin{lstlisting}[language=bash]
git add nunchuk.c
git add Makefile
git commit -m "laboratorijska vjezba 5 gotova"
git push origin master
\end{lstlisting}
\end{document}
