\documentclass[11pt]{article}
\usepackage[croatian]{babel}  % Croatian typographical rules and hyphenation patterns
\usepackage[utf8]{inputenc}  	% Encoding of Croatian characters
\usepackage[T1]{fontenc}
\usepackage{ae,aecompl}     	% Type 1 fonts, similar to Computer Modern

\usepackage{microtype}				% Improves spacing

\usepackage{booktabs}					% Nice looking tables
\usepackage{enumerate}				% Additional options for listing of items in enumerate environment
\usepackage{algorithm2e}			% Writing pseudo-code
\usepackage{todonotes}				% Adding todo items
\usepackage{dirtree}					% Simple display of directory tree
\usepackage{hyperref}
\hypersetup{
    colorlinks=true,
    linkcolor=blue,
    filecolor=magenta,
    urlcolor=cyan,
}
\urlstyle{same}
\usepackage{graphics}
\title{
	\Large Sveučilište Josipa Jurja Strossmayera u Osijeku \\
	Fakultet elektrotehnike, računarstva i informacijskih tehnologija \\
	\vspace{4cm}
	\Large Kolegij: Linux u ugradbenim sustavima \\
	\vspace{4cm}
	\Large \textbf{Laboratorijska vježba 1: Uvod u Linux okruženje}
	}
\date{}
\begin{document}
\maketitle
\thispagestyle{empty}
\newpage

\section{Uvod}
Ciljevi vježbe: Upoznavanje s radom u terminalu. Rad s upraviteljem paketa.
Rad s git sustavom za kontrolu verzije dokumenata. Struktura OS-a i manipulacija datotekama. Korištenje tekstualnog
editora. Prevođenje i pokretanje programa. Korištenje mrežnih protokola.

\section{Pokretanje operacijskog sustava. Pokretanje terminala}
Nakon uključivanja računala u izborniku \textit{boot} sustava izaberite
\textit{Ubuntu}. Korisničko ime je \textit{rtrk}, a lozinka \textit{rtrk}.
Pričekajte da se pokrene operacijski sustav. Na ekranu će se pojaviti
grafičko sučelje. Upoznajte se s grafičkim sučeljem. Primjerice, pokušajte
pronaći status Ethernet veze vašeg računala.
\newline
\newline
Pritiskom na tipku \textit{Super} (tipka lijevo od \textit{Alt}, tzv.
\textit{Windows} tipka) otvara se \textit{Search} u koji zatim upišete
 \textit{terminal} i pritisnite tipku enter. Pokrenit će se terminal u kojem je
 aktivan naredbeni redak, tj. moguće je upisivati naredbe u njega i pritiskom
 na tipku \textit{Enter} započeti izvršavanje unešene naredbe. Naredbe se
 izvršavaju uz pomoć \textit{bash} ljuske (engl. \textit{shell}). U isto
 vrijeme moguće je imati i više pokrenutih terminala (korisno u mnogim
 situacijama). Nakon pokretanja terminala, naredbena ljuska je pozicionirana
 u vašem \textit{home} direktoriju. Provjerite ovo upisom naredbe \textit{pwd}.
\newline
\newline
\textbf{Napomena}: Prilikom pisanja naredbi koristite tipku \textit{Tab} za
 dovršavanje naredbi, putanja, imena datoteka i sl. jer ćete na taj način
 izbjeći moguće sintaksne pogreške.
 \section{Korištenje upravitelja paketa}
Registrirajte se na stranici \url{gitlab.com} (preporuča se korisničko ime
 oblika \textit{imeprezime} ili \textit{xprezime} pri čemu je \textit{x} prvo
 slovo imena). Po potrebi napravite osnovnu konfiguraciju računa (npr.
 kreiranje \textit{SSH} ključeva). Napravite \textit{fork}
 \textit{embedded\_linux} projekta na adresi
 \url{https://gitlab.com/rgrbic/embedded\_linux.git}.
 \newline
 \newline
 Klonirajte repozitorij \textit{embedded\_linux} na PC računalo koristeći
 naredbu \textit{git clone}. Napravite granu \textit{rad\_na\_vjezbi} pomoću
 naredbe \textit{git branch –b} i odmah se prebacite u nju.
 \section{Kretanje kroz direktorije i prikaz sadržaja direktorija}
Sadržaj home direktorija izlistajte naredbom \textit{ls}. Pomoću naredbe
 \textit{man ls} provjerite koje su sve dostupne opcije ove naredbe. Izlistajte
 sadržaj direktorija pomoću opcija \textit{-l}, \textit{-a}. Kombiniraje obje
 opcije u istoj naredbi. Što pružaju ove opcije prilikom prikaza sadržaja
 direktorija? Kako se označavaju direktoriji, a kako datoteke? Što predstavlja
 . ispred imena?
\newline
\newline
Pozicionirajte se u \textit{/home/rtrk/embedded\_linux/LV1/resources}
 direktorij. Izlistajte samo datoteke/direktorije koji počinju s malim slovom.
 Izlistajte sve datoteke koje završavaju s \textit{.txt}. Izlistajte sve
 datoteke koje ne završavaju s \textit{.txt}.
\newline
\newline
Pomoću naredbe \textit{cd} pozicionirajte se u korijenski direktorij. Na koje
 sve načine to možete učiniti? Prikažite sadržaj korijenskog direktorija.
 Proučite dobivenu strukturu sustava datoteke.
\newline
\newline
\textit{Linux} sve uređaje vidi kao datoteke. U kojem direktoriju se nalaze
 uređaji? Iz svog home direktorija, koristeći apsolutnu putanju, prikažite
 detaljan sadržaj tog direktorija. Prikažite samo uređaje/datoteke koje počinju
 s malim slovom "d". Možete li zaključiti što predstavljaju izlistane datoteke?
 \section{Manipulacija datotekama}
 Pozicionirajte se u \textit{/home/rtrk/embedded\_linux/LV1/resources}
 direktorij. Kopirajte sve datoteke koje u svom imenu sadrže znak 9 u
 direktorij \\ \textit{/home/rtrk/embedded\_linux/LV1/resources/test9}.
 Izbrišite sve datoteke u \textit{/home/rtrk/embedded\_linux/LV1/resources}
 koje u svom imenu sadrže znak "9". Provjerite rezultate izlistavanjem sadržaja
 direktorija.
\newline
\newline
Napravite novi direktorij u \textit{/home/rtrk/embedded\_linux/LV1/resources}
 naziva \textit{Test}. Kopirajte sve datoteke koje imaju u imenu niz
 \textit{test\_datoteka} u direktorij \textit{Test}. Promijenite naziv
 direktorija \textit{Test} u \textit{test}. Obrišite direktorij \textit{test}.
 Provjerite rezultate izlistavanjem sadržaja direktorija.
 \newline
 \newline
Kopirajte direktorij \textit{/home/rtrk/embedded\_linux/LV1} u \\
 \textit{/home/rtrk/embedded\_linux/LV1\_temp}. Provjerite rezultate
 izlistavanjem sadržaja direktorija. Obrišite direktorij \textit{LV1\_temp}.
\newline
\newline
Izbrišite sve datoteke koje imaju u imenu niz \textit{test\_datoteka} u
 direktoriju \textit{/home/rtrk/embedded\_linux/LV1/resources}.
\newline
\newline
Raspakirajte datoteku
 \textit{/home/rtrk/embedded\_linux/LV1/resources/mbox.txt.bz2}. Spremite
 stanje repozitorija u lokalnu bazu kao i na udaljeno računalo pomoću
 \textit{git} naredbi \textit{commit} i \textit{push}. Prilikom spremanja
 stanja pomoću \textit{commit} naredbe napišite odgovarajuću poruku.

\section{Prikaz sadržaja datoteka. Editiranje datoteka. Prava}
Prikažite sadržaj datoteke \textit{mbox.txt} pomoću naredbi \textit{cat},
 \textit{more} i \textit{less}. Prikažite prvih 20 odnosno posljednjih 20
 redaka ove datoteke. Po potrebi promijenite prava nad datotekom
 \textit{mbox.txt}.
\newline
\newline
Prikažite sve email adrese koje se pojavljuju u datoteci
 \textit{mbox-short.txt}. Spremite sve pronađene email adrese u datoteku
 \textit{emails.txt}. Promijenite prava datoteke \textit{emails.txt} tako da je
 moguće samo čitanje datoteke od svih korisnika.
\newline
\newline
Pronađite email adrese u datoteci \textit{mbox.txt} i spremite ih u
 \textit{emails1.txt}. Usporedite datoteke \textit{emails.txt} i
 \textit{emails1.txt}. Potražite brojeve u svakoj datoteci u direktoriju \\
 \textit{/home/rtrk/embedded\_linux/LV1/resources/test9} pri čemu rezultat
 treba prikazati na zaslonu računala. Rezultat trebaju biti brojevi 9, 19, 29,
 39… (svaki broj je u zasebnom redu). Zapišite u datoteku \textit{brojevi.txt}
 rezultate pretraživanja (bez naziva izvorne datoteke). Prikažite sadržaj
 datoteke \textit{brojevi.txt}. Obrišite direktorij \textit{test9}.
\newline
\newline
Pokrenite nano tekstualni editor. Otvorite datoteku \textit{emails.txt}.
 Zamijenite znak \textit{@} s praznim znakom.
\newline
\newline
Spremite stanje repozitorija u lokalnu bazu kao i na udaljeno računalo pomoću
 \textit{git} naredbi \textit{commit} i \textit{push}. Prilikom spremanja
 stanja pomoću \textit{commit} naredbe napišite odgovarajuću poruku.
 \section{Prevođenje i pokretanje programa}
 Prevedite programe \textit{cipher\_encryption.c} i
 \textit{cipher\_decryption.c} pomoću \textit{gcc} na način da se generiraju
 izvršne datoteke \textit{enrypt} i \textit{decrypt}. Šifrirajte poruku
 „Uvod u Linux“ pomoću generirane izvršne datoteke \textit{encrypt}. Koristite
 ključ 8. Dešifrirajte poruku pomoću \textit{decrypt} izvršne datoteke.
 Izmijenite izvorni kod tako \textit{cipher\_encryption.c} da se rezultat
 sprema u datoteku pod nazivom \textit{poruka.txt} odnosno da program
 \textit{cipher\_decryption.c} čita i dešifrira tekst iz datoteke
 \textit{poruka.txt}. Prevedite promijenjeni izvorni kod.
\newline
\newline
Pokušajte prevesti izvorni kod s uključenom opcijom \textit{-Wall}. Zašto
 \textit{C} funkcija \textit{gets} nije sigurna? Kako biste riješili ovaj
 problem?
\newline
\newline
Za sve generirane datoteke kao i izvorni kod promijenite prava tako da ih
 ostali korisnici ne mogu mijenjati, čitati niti izvršavati.
\newline
\newline
Spremite stanje repozitorija u lokalnu bazu kao i na udaljeno računalo pomoću
 \textit{git} naredbi \textit{commit} i \textit{push}. Prilikom spremanja
 stanja pomoću \textit{commit} naredbe napišite odgovarajuću poruku.

 \section{Dohvaćanje datoteka pomoću \textit{wget}. \textit{SSH} komunikacija}
Preuzmite \textit{HTML} datoteku s \textit{URL}-a \\
\url{https://www.ferit.unios.hr/studiji/sveucilisni-diplomski-studij} pomoću
 naredbe \textit{wget}. Pomoću naredbe \textit{grep} prikažite sve redove u
 datoteci koji u sebi imaju riječ „programiranje“. Ako naredba ne vraća nikakav
 rezultat, provjerite pomoću \textit{man grep} kako čitati binarne datoteke kao
 da su tekst.
\newline
\newline
Naredna vježba radi se u paru. Potrebno je ostvariti \textit{SSH} vezu između
 dva računala. Na jednom računalu potrebno je instalirati
 \textit{openssh-server}. Na istom računalu potrebno je kreirati novog
 korisnika. S drugog računala spojiti se \textit{SSH} vezom na računalo na
 kojem se nalazi \textit{SSH} server pomoću podataka za kreiranog korisnika.
 Pokretanjem programa \textit{encrpyt} koji se nalazi u
 \textit{/home/rtrk/embedded\_linux/LV1/resources/} ostaviti šifriranu poruku
 korisniku \textit{rtrk} u datoteci \textit{poruka.txt}. Usmeno razmijeniti
 ključ za šifriranje. Korisnik \textit{rtrk} na serveru treba dešifrirati
 poruku.

\section{Spremanje rada}
Vratite se u glavnu granu git sustava pomoću naredbe
 \textit{git checkout master}. Sve naredbe koje ste unosili tijekom rada na
 vježbi spremite u odgovarajuću datoteku pomoću naredbe:
\newline
\newline
\textit{history > <vase\_ime\_prezime>.txt} (npr.
\textit{history > pero\_peric.txt})
\newline
\newline
Datoteku prebacite u \textit{/home/rtrk/embedded\_linux/LV1/solutions/}.
 Spremite promjene u lokani repozitorij pomoću naredbe \textit{git commit} kao
 i na vaš udaljeni račun pomoću \textit{git push –u origin master}.Napravite
 „merge request“ na \textit{Gitlab} stranici kako bi se vaša datoteka
 dodala u izvorni repozitorij \textit{embedded\_linux}.
 \section{Korisne naredbe}
Sljedeće naredbe su korisne i potrebno ih je znati: \textit{ls, grep, pwd, cd,
 mkdir, rm, cp, apt-get, diff, git push, git commit, git checkout, git pull,
 cat, more, less, nano, chown}.

\end{document}
